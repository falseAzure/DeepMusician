%% Document Type
\documentclass[xcolor={dvipsnames},12pt]{beamer}

%% Packages
% Language
\usepackage[greek,french,ngerman, english]{babel} % Change language \selectlanguage{languageA}. Last listed language defines document


%% Different font and encoding
\usepackage[LGR,T1]{fontenc} % Extended Cork (EC) fonts in T1 encoding contains letters and punctuation characters for most of the European languages using Latin script. see: https://tex.stackexchange.com/questions/664/why-should-i-use-usepackaget1fontenc
\usepackage{lmodern} % Different font: better for accents and ß
\usepackage[utf8]{inputenc} % Manages the input. The utf8 encoding used by inputenc only defines the characters that are actually provided by the fonts used.
\usepackage{tgbonum} % More fonts
\usepackage{tgcursor} % Even more fonts
\usepackage{fontspec} % Package for fonts
% Graphics
\usepackage{graphicx}
\graphicspath{ {img/} } % set the folder of the images
% \includegraphics[options]{overleaf-logo}
\usepackage{wrapfig} % Produces figures which text can flow around
\usepackage{subfig} % For subfigures

%% Highlighting
% xcolor is loaded per default with Beamer
% \textcolor{color}{text}, \colorbox{color}{text}

% define Colors
\definecolor{links}{HTML}{157394}
% \definecolor{links}{HTML}{03588C}
\definecolor{title_bg}{HTML}{199988}
% \definecolor{title_bg}{HTML}{D9414E}
% \definecolor{title_bg}{HTML}{DC4C4C}
\definecolor{title_text}{HTML}{FFFAFA}
\definecolor{background}{HTML}{FFFAFA}
\definecolor{bar}{HTML}{199988}
% 201E50


% References
\usepackage{hyperref}
% e.g.: \href{}{}
\usepackage{cleveref} % more complex package to manage references - only works with numbered items
% e.g. \label{itm:d1} > \cref{itm:d1}
\hypersetup{
    colorlinks, linkcolor = ,urlcolor = links
}


% Complex and long tables
% https://tex.stackexchange.com/questions/35293/p-m-and-b-columns-in-tables
% https://de.wikibooks.org/wiki/LaTeX-W%C3%B6rterbuch:_tabular
\usepackage{longtable} % tabular for tables across multiple pages
\usepackage{tabularx} % tabulars with flexible column width
\usepackage{multicol} % multicolumn environment
\usepackage{multirow} % multirow environment
\usepackage{booktabs} % better looking tables
\newcommand{\tabitem}{~~\llap{\textbullet}~~} % Lists without environment

% Design trees and forests
\usepackage{tikz, tikz-qtree, tikz-qtree-compat} % https://www.bu.edu/math/files/2013/08/tikzpgfmanual.pdf
\usetikzlibrary{positioning}
\usepackage{forest}
\usepackage{mdframed} % framed environments that can split at page boundaries

% deprecated
% \usepackage{tikzit}
% \input{tikz_style.tikzstyles}

% Packages to represent Code 
% for algorithms
\usepackage{algorithm}
\usepackage{algpseudocode}

% continuation indent patch for algorithms
\newlength{\continueindent}
\setlength{\continueindent}{2em}
\usepackage{etoolbox}
\makeatletter
\newcommand*{\ALG@customparshape}{\parshape 2 \leftmargin \linewidth \dimexpr\ALG@tlm+\continueindent\relax \dimexpr\linewidth+\leftmargin-\ALG@tlm-\continueindent\relax}
\apptocmd{\ALG@beginblock}{\ALG@customparshape}{}{\errmessage{failed to patch}}

% another packages for algorithms
\usepackage{listings}
\lstset{
  language=R,
  tabsize=3,
  basicstyle=\small,
  breaklines=true,
  basicstyle=\footnotesize\ttfamily,
  breaklines=true
}
% e.g. \begin{lstlisting} Code \end{lstlisting}


% packages for Math
\usepackage{amsmath, amssymb, enumerate}
\usepackage[donotfixamsmathbugs]{mathtools}


%% Settings
% Theme = metropolis
\usetheme[titleformat=smallcaps, sectionpage=progressbar, subsectionpage = progressbar, numbering=fraction, block=fill, progressbar=none, titleformat frame=smallcaps, titleformat section=smallcaps]{metropolis}


% Footline
\makeatletter
\setbeamertemplate{footline}{%
    \begin{beamercolorbox}[colsep=1.5pt]{upper separation line head}
    \end{beamercolorbox}
    \begin{beamercolorbox}[wd=\paperwidth]{section in head/foot}
        

        \insertsubsectionnavigationhorizontal{\textwidth}{}{\hskip10pt plus1filll
        \usebeamertemplate*{frame numbering}\hskip7pt}\vskip3pt%
    \end{beamercolorbox}%
    \begin{beamercolorbox}[colsep=1.5pt]{lower separation line head}
    \end{beamercolorbox}
}
\makeatother

% Captions (needs subfig)
\captionsetup{justification=raggedright,singlelinecheck=false}

% Colors
\setbeamercolor{progress bar}{fg=bar}
\setbeamercolor{frametitle}{fg=title_text, bg=title_bg}
\setbeamercolor{section in head/foot}{fg=title_text, bg=title_bg}
\setbeamercolor{subsection in head/foot}{fg=title_text, bg=title_bg}
\setbeamercolor{background canvas}{bg=background}
% \setbeamercolor*{structure}{bg=background, fg=background}

% Fonts
\setmonofont{Courier New}[ 
  Scale=1.1
]
\setbeamerfont{section title}{series=\mdseries}
\setbeamerfont{frametitle}{series=\mdseries}
\setbeamerfont{title}{series=\mdseries}

% ToC
\AtBeginSection{
    \begingroup
    \setbeamertemplate{footline}{}
    \begin{frame}
        \sectionpage
        \begin{multicols}{3}
        \tableofcontents[currentsection]
        \end{multicols}
    \end{frame}
    \endgroup
}


% Metadata 
\title{Deep Musician}
\subtitle{Automatic Music Generation using DeepLearning}

\author{Fabian Gabelberger}


% Document
\begin{document}
\maketitle

\begin{frame}{Idea}
    \textbf{Music Generation}
    \begin{itemize}
        \setlength\itemsep{1em}
        \item Create a model that can automatically generate music
        \item Train the model on a large dataset of existing music
        \item Use the model to generate new, unheard music of arbitrary length
    \end{itemize}
\end{frame}

\begin{frame}{Approach}
    \textbf{Preprocessing - get the data in shape}
    \begin{itemize}
        \item Use symbolic representation of music
        \item[$\rightarrow$] Midi Files
        \item[$\rightarrow$] \textbf{Piano roll}, that captures the
            notes played at each time step (2D-Array)
    \end{itemize}

    \textbf{Architecture}
    \begin{itemize}
        \item Sequence aware \textbf{encoder-decoder} architecture
        \item Each consist of a two-layered \textbf{GRU}
        \item The \textbf{encoder} encodes the input sequence at once
        \item The \textbf{decoder} generates the output sequence step by step according
              to the previous ones
    \end{itemize}
\end{frame}

\begin{frame}{Generation}
    \textbf{Sequence 2 Sequence}
    \begin{itemize}
        \item During \textbf{training} the model tries to predict the next note
        \item After training the model is started with an empty sequence
        \item[$\rightarrow$] This triggers the model to incrementally
            \textbf{generate}
            a sequence of notes
    \end{itemize}
\end{frame}

\begin{frame}{Results}
    \begin{itemize}
        \item The results measured by the loss function are very promising.
        \item Yet the generated music is not very good and most of the time
              only an empty sequence is returned.
        \item[$\rightarrow$] The model is not able to generate a coherent
            melody, since the training data is very \textbf{imbalanced}: Most
            of the time the notes are not played at all.
    \end{itemize}
\end{frame}

\begin{frame}{Insights}
    \textbf{Preprocessing}
    \begin{itemize}
        \item Endless possibilities to represent the underlying data
        \item There is not a single best representation: each have there own
              advantages and disadvantages
        \item Beside the piano roll representation, there are also symbolic
              representations that represent the music as a sequence of words.
    \end{itemize}
\end{frame}

\begin{frame}{Insights}
    \textbf{Evaluation}
    \begin{itemize}
        \item Although the loss function is a good indicator for the
              performance of the model, it is not a good measure for the
              quality of the generated music.
        \item Instead of a classical BCE-loss, I used a loss function that was
              initially developed for image recognition and object detection:
              \textbf{Focal Loss}.
        \item[$\rightarrow$] The main idea behind focal loss is to down-weight
            the contribution of easy examples in the training data and focus more
            on the hard examples
    \end{itemize}
\end{frame}

\end{document}