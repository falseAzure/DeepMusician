%% Document Type
\documentclass[a4paper, 10pt, xcolor=dvipsnames]{article} % use option titlepage to get the title on a page of its own.
\pagenumbering{arabic} % define page numbering. [gobble] omits any numbering.

%% Packages
% Document Layout
\usepackage[left=3cm,right=3cm,top=2cm,bottom=2cm]{geometry}   % to change margins
\usepackage{titling}             % Uncomment both to   
\setlength{\droptitle}{-2cm}     % change title position 
\date{\vspace{-9ex}} % delete date - comment to insert date

% Language
\usepackage[greek,french,ngerman, english]{babel} % Change language \selectlanguage{languageA}. Last listed language defines document


% Different font and encoding
% \usepackage[LGR,T1]{fontenc} % Extended Cork (EC) fonts in T1 encoding contains letters and punctuation characters for most of the European languages using Latin script. see: https://tex.stackexchange.com/questions/664/why-should-i-use-usepackaget1fontenc
\usepackage{lmodern} % Different font: better for accents and ß
\usepackage[utf8]{inputenc} % Manages the input. The utf8 encoding used by inputenc only defines the characters that are actually provided by the fonts used.
% \usepackage{tgbonum} % More fonts

% Bibliography
\usepackage[style=numeric-comp, sorting=none, url=true, eprint=false, isbn=false, backend=biber, citetracker=true]{biblatex} % biblatex package handling bibliography
\bibliography{../../../../../Bibliography/bibliography.bib} % path of bibliography
\usepackage[autostyle=tryonce]{csquotes} % used for quotes: use either /textquote{} or /blockcquote

% Graphics
\usepackage{graphicx}
\graphicspath{ {img/} } % set the folder of the images
% \includegraphics[options]{overleaf-logo}
\usepackage{subcaption} % group images > Support for sub-captions
\usepackage{wrapfig} % Produces figures which text can flow around


% Caption of Floats (images, tables etc)
\usepackage{caption}
\captionsetup{justification   = raggedright,
              singlelinecheck = false}
% control space before and after image/table
\setlength{\belowcaptionskip}{-20pt}
\captionsetup{belowskip=0pt}


% Highlighting
\usepackage[dvipsnames]{xcolor} % for colors see: https://www.namsu.de/Extra/pakete/Xcolor.html and https://en.wikibooks.org/wiki/LaTeX/Colors 
% \textcolor{color}{text}, \colorbox{color}{text}


% References
\usepackage{hyperref}
% e.g.: \href{}{}
\usepackage{cleveref} % more complex package to manage references - only works with numbered items
% e.g. \label{itm:d1} > \cref{itm:d1}
\hypersetup{
    colorlinks = false,
    linkbordercolor = {OliveGreen}
}


% Complex and long tables
% https://tex.stackexchange.com/questions/35293/p-m-and-b-columns-in-tables
% https://de.wikibooks.org/wiki/LaTeX-W%C3%B6rterbuch:_tabular
\usepackage{longtable} % tabular for tables across multiple pages
\usepackage{tabularx} % tabulars with flexible column width
\usepackage{multicol} % multicolumn environment
\usepackage{booktabs} % better looking tables


% Design trees and forests
\usepackage{tikz-qtree, tikz-qtree-compat} % https://www.bu.edu/math/files/2013/08/tikzpgfmanual.pdf
\usetikzlibrary{positioning}
\usepackage{forest}
\usepackage{mdframed} % framed environments that can split at page boundaries


% Packages to represent Code 
% for algorithms
\usepackage{algorithm}
\usepackage{algpseudocode}


% continuation indent patch for algorithms
\newlength{\continueindent}
\setlength{\continueindent}{2em}
\usepackage{etoolbox}
\makeatletter
\newcommand*{\ALG@customparshape}{\parshape 2 \leftmargin \linewidth \dimexpr\ALG@tlm+\continueindent\relax \dimexpr\linewidth+\leftmargin-\ALG@tlm-\continueindent\relax}
\apptocmd{\ALG@beginblock}{\ALG@customparshape}{}{\errmessage{failed to patch}}


% another packages for algorithms
\usepackage{listings}
\lstset{
  language=SQL,
  tabsize=3,
  basicstyle=\small,
  breaklines=true
}
% e.g. \begin{lstlisting} Code \end{lstlisting}

% Abstract
\renewenvironment{abstract}
 {\small
  \begin{center}
  \bfseries \abstractname\vspace{-.5em}\vspace{0pt}
  \end{center}
  \list{}{
    \setlength{\leftmargin}{3cm}%
    \setlength{\rightmargin}{\leftmargin}%
  }%
  \item\relax}
 {\endlist}


%% Metadata 
\title{Deep Musician
\vspace{0.25em}
\\ \large Automatic Music Generation using Deep Learning
}
\author{Fabian Gabelberger}
\date{\today}


%% Document
\begin{document}
\maketitle
\vspace{2em}

\begin{abstract}
  The main idea of this project is to create a model that is capable of
  automatically generating new and unheard musical melodies that resemble human
  ones. This is achieved with a deep neural network that was trained with a
  large amount existing MIDI files. The network uses a sequence aware
  Encoder-Decoder structure that is capable of creating sequences of notes of
  arbitrary length. The encoder and decoder each consist of a 2-layer GRU
  network, whereas the decoder has an additional classifier.
\end{abstract}

\section{Introduction}
With \emph{MuseNet OpenAI} created a deep neural network that
\blockcquote{payne2019}{can generate 4-minute musical compositions with 10
  different instruments, and can combine styles from country to Mozart to the
  Beatles.} Behind this project resides the (philosophical) idea that musical
compositions can arise not only from a particular (abstract) artistic
understanding of harmony, rhythm, melody, etc., but also \emph{solely} from a
variety of previous works that are incorporated into the new, unheard piece as
a wealth of experience.

MuseNet is fuelled by \blockcquote{payne2019}{"a large-scale \emph{transformer
    model} trained to predict the next note(s) in a sequence.} While this
approach using a transformer model is certainly state of the art for
sequential data and produces a truly vibrant and rich musical style, it is
very resource intensive and can thus -- simply due to hardware limitations
-- not be the scope of a project for \emph{Applied} Deep Learning .

In contrast, there are also more lightweight approaches that use a variety of
RNN structures (such as LSTM models or RNN \cite{lewandowski2012,
  jedrzejewska2018, hewahi2019, ycart2017, mangal2019} with Self-Attention
\cite{jagannathan2022} and even a CNN \cite{yang2017} -- which is quite
interesting considering the temporal dimension of music that CNN are not
designed to depict. (For a general overview of the different approaches of
generative music models see: \cite{briot2017}.)

\section{The Problem}
\subsection{A philosophical Problem?}

All these ideas try to solve the problem of how to outsource the creation of
music -- which until now has always been in the hands of man -- to the machine.
In terms of music history, this is not necessarily about creating original
works that represent a new avant-garde in music. From a purely pragmatic point
of view, such a model could rather offer musicians the possibility to
complement their works (be it with certain instruments or as a source of ideas)
or to generally grant artists access to freely available and and
non-proprietary music in order to easily create background music for a video or
other content, for example.

At the bottom of this question itself, lies the problem of the creative and
generative abilities of machines, algorithms, models etc. Since machines are
said to produce merely repetitive patterns, it is hard to imagine how true
creative output is to emerge from them. Yet the recent successes of deep
(reinforcement) neural models\footnote{The creative/machine generation field
  has gained tremendous traction through its successes in recent years, from
  Deepmind's AlphaZero, which plays chess, Go, and Shogi, to various image
  processing and generation software, to OpenAI's latest ChatBot. The list is
  constantly being refreshed and expanded, and it seems we've only just exhausted
  the first stages of machines' creative potential.} has shown, that the boundary
between human and machine creation and creativity is getting ever thinner and
thinner.\footnote{Especially if we acknowledge that the boundary between
  repetition and creativity is not a strict criterion, but a fluid line, where
  both sides are mutually dependent.}

While I neither address nor answer these philosophical questions on a
theoretical level in the project, I will make a first delicate attempt to
explore machine-creative ways by means of a practical application. So the point
is not to find a theoretical answer to the question of creativity, but to show
by means of a practical application to what extent machine learning produces
musical pieces that we as humans grasp as melodious, beautiful, and and
ultimately human. Thus reformulating the former existential question to a
rather categorical and technical one: How do the creations of machines differ
from that of their human "counterparts" -- and are we able to tell the
difference?

\subsection{The concrete problem}

Concretely, i.e. looking at the practical application and thus at the field of
computer science and artificial intelligence, the problem that such an
algorithm needs to tackle is that it not only needs to make a prediction about
the further course of a musical piece, but to design such a piece itself. While
a large part of neural networks is trained to classify different things or to
deliver the corresponding output for a given input (be it a classification, a
translation or similar), this application is about \emph{creating} something
("without" a corresponding input). In addition, musical pieces are sequences
that have a temporal component at their core, i.e. their elements build on each
other in time and do not exist synchronously or simultaneously in the moment.

Two important elements that we therefore have to take into account when dealing
with the topic of machine-generated music are:
\begin{enumerate}
  \item Sequence / temporal awareness
  \item Generation instead of classification
\end{enumerate}


\section{The Solution}
The first point is solved by certain neural architectures (in particular RNN,
LSTM and Attention Mechanism) that allow to integrate a temporal structure into
the machine learning process. As a result, the individual training data are no
longer (temporally) independent data points, but become a coherent sequence
whose course has to be traced, i.e. learned, by the algorithm. The algorithm is
given the steps in chronological order and its task is to correctly predict the
next step in time. During training, the model learns to correctly continue the
given sequences.

The second step is approached by not continuing existing melodies correctly
after training, but by animating the model to predict new (own) melodies. This
is done by passing an empty input to the model on the basis of which the
network predicts the temporally following second element and on the basis of
these two elements the third and so on.


In the introduction we have heard about different and very potent architectures
that are sequence aware and capable of generation. Although I would have loved
to implement an Attention Mechanism, I decided to start on a much smaller scale
with a rather simple encoder-decoder architecture, that each consist auf a
two-layered GRU structure. My project is thus dedicated to a small section of
the topic of creative algorithms, namely the automatic generation of music, by
implementing a sequence aware model, that is capable of generating musical
sequences of arbitrary length.\footnote{I realize that I am only scratching the
  surface of the issue given the incredibly vast and comprehensive models.}

Generative algorithms that attempt to perform an artistic feat are almost
exclusively implemented as neural networks. This is simply because they are the
strongest and most comprehensive algorithms and can produce new data that did
not occur during their training. Of course, it would also be possible to
generate melodies via rule-based algorithms. However, these have the
disadvantage of always following a certain rule or pattern and do not manage to
surprise the user. The melody will always run within a certain framework, but
will never have a "creative" potential beyond what has already been seen.
Moreover, previous experiments in the literature have shown that the approach
of creating generative melodies using neural networks is very successful. Of
course, it would also be exciting to implement this reinforcement learning and
to train the algorithm via positive and negative incentives. However, the
problem here is that we would need a measurable criterion for beautiful and
harmonic melodies. But to be frank, due to hardware limitations, the melodies
my model produced so far are similar to simple based melodies, which are not
very interesting. Yet, this should be solved with more powerful hardware.

\section{Outline of the Project}
\subsection{Goal}
The goal of the project is to create a model that produces melodies that sound
human, or natural, i.e., not mechanical. Making this goal mathematically
quantifiable is not trivial, since there are a variety of ways to represent
notes and sequences of notes that each require different metrics. In addition
most of these metrics are not able to capture the essence of a creative
sequence of notes. Of course, there are metrics that describe how well an
algorithm predicts a certain sequence, but as we will see below, these have
weighty drawbacks. Therefore, I remain with the approach of measuring generated
melodies according to simply my human ear.

\subsection{Data Structure}

\subsubsection{Representing Music}

The basis of music is formed by sequentially played sounds or tones that can
be represented as a complex \emph{waveform}. These individual sounds can be joined
together in any way to form an entire piece of music, which in turn is again a
single waveform that we can play back and listen to in different audio
formats (MP3, FLAC, WAV etc.).

\subsubsection{MIDI}

While this form of representation already depicts a concrete shaping of the
music in the form of an unique audio file, it is also possible to specify the
individual tones of the piece in the form of notes with different parameters.
The advantage here is that the concrete instrumentation is abstracted from and
only the internal \emph{structure} of the piece is considered. The generally
accepted standard for this representation is \emph{MIDI}. By means of MIDI it
is possible to transmit not only the pitch and length of the individual notes,
but also other parameters such as velocity -- yet, no concrete waveform is
produced.

Thus, due to its abstract nature MIDI offers the possibility to extend the
input of the model successively. While initially only monophonic audio tracks
with constant dynamics and tone length are used, these parameters are to be
successively added to the input to see how the created melodies change.

\subsubsection{Preprocessing}
However, neural networks cannot be trained with midi files themselves.
Therefore, these must be converted into another form in which the notes can be
passed to the network. There are a lot of possibilities for this, of which I
have chosen a rather classical and simple one: The piano roll. Here a 2
dimensional matrix is spanned, whose x-axis represents the time and whose
y-axis represents the 88 notes of the piano. Each touch of a note is marked
with 1 in the matrix at the corresponding time t -- all other cells remain empty
(0). This representation is very clear and intuitive. However, it has a big
problem: since only a small percentage of the available cells are filled --
i.e. most of the time NO note is played -- there is a big imbalance in the
data. Here other representations such as the language based ones would have
definitely been better.

\subsection{Metric and Loss}
Although the final goal of the model is to generate melodies that sound human,
we need to define a metric and loss that quantifies this. This is not trivial,
since classical metrics have difficulty dealing with this problem and return
suggestive and misleading values and classical losses do not optimise for the
desired goal, that is the generation of a human sounding melody.

I faced this problem during the later stages of my experiment, when the model
easily learned according to the classical BCE-loss, but afterwards during
testing only generated empty melodies. This was due to the fact, that it
actually guessed almost all of the notes correctly as they were not being
played. So the empty sequence resembled the input it was given most of the
time. Or put, differently the model was stuck in a local minimum.

\subsubsection{Focal loss}

I learned that image classification faces a similar problem and solves this by
using a so called focal loss. Focal loss is a loss function that is used in
image classification tasks, particularly those involving object detection. The
main idea behind focal loss is to down-weight the contribution of easy examples
in the training data and focus more on the hard examples, which are typically
the ones that are more challenging to classify correctly. This is achieved by
modifying the standard cross-entropy loss function by introducing a weighting
term that increases the loss for easy examples and decreases the loss for hard
examples. The result is a loss function that is more "focal" on the hard
examples and helps the model to better learn from them and improve its
performance on the task.

With the introduction of focal loss in my model it started generating
meaningful sequences of notes. To keep track of the validity of the generation
of sequences I also introduced a density metric, that measures the average
notes played per time step. What is more, lowering the learning rate also
helped a lot in not getting stuck in a local minimum.

Yet, the two parameters of the focal loss (alpha and gamma) need to be
carefully adjusted to obtain meaningful results, which for this project has not
been accomplished so far.


\subsection{Architecture}

The architecture of the network consists of an encoder-decoder structure and a
subsequent classifier. The input (a piano roll with length 96) is first passed
through the encoder -- only the returned hidden state is of importance here,
the output itself is discarded. Then an empty input (SOS-token) with the hidden
state is passed to the decoder. This gives a new hidden state and a prediction
about the next state. This prediction is then passed through the classifier and
serves as a new input for the decoder together with the updated hidden state.
This process runs over the entire length of the sequence, meaning that we
iterate through the sequence length by starting with the SOS-Token, then using
the previous output and hidden state as new input and. So while the encoder
processes the incoming sequence all at once, the decoder and classifier build
up the new sequence step by step.

\begin{itemize}
  \item EncoderRNN: GRU(input: 88, hidden: 512, layers=2, dropout=0.2)
  \item DecoderRNN: GRU(input: 88, hidden: 512, layers=2, dropout=0.2)
  \item Classifier
        \begin{itemize}
          \item  Linear(in\_features=512, out\_features=256, bias=True)
          \item  ReLU
          \item  Dropout(p=0.5)
          \item  Linear(in\_features=256, out\_features=88, bias=True)
          \item  Sigmoid
        \end{itemize}
\end{itemize}

\section{Main Takeaways}

The three most important takeaways of the project for me concern preprocessing,
the evaluation and hardware.

\subsection{Preprocessing}
Here I learned in particular that the preprocessing part can
exceed the other parts of the project by a wide margin in terms of time. So far
I have worked in the field of machine learning especially with text,
audiovisual data and numerical values. However, the symbolic representation of
music was previously unknown to me from a computer science perspective. While I
know that preprocessing from the other domains can also be time consuming, I
was just overwhelmed by the wealth of possibilities with symbolic music and the
time I needed to invest in order to obtain a meaningful representation of the
music. Here it was important to keep a good overview and not get lost in the
possibilities. The individual advantages and disadvantages may not be evident
at first, but if you start with a simple approach, you can slowly work your way
up and document the differences and improvements. It is important to keep in
mind, however, that the different forms of representation are often so
different that they cannot simply be fed into the existing network -- for
example, if one chooses a text-based representation as opposed to a piano roll.

Linked to this is also the insight that there can be an unbelievable number of
approaches to solving a problem, which already begin during preprocessing and
not just during modeling. One often has the feeling that the differences only
become apparent or noticeable during modeling. However, I learned that it can
be important to always take a step back and look at the problem with new eyes
and possibly even change the underlying data structure to achieve your goal. It
was really impressive to see the sheer number of ways that symbolic music can
be represented (from classical piano rolls to linguistic models), all of which
highlight different aspects and provide benefits. I would have liked to try out
more of these approaches -- especially, since the language based ones would
have inherently solved the problem of imbalanced data. But in the end I was
glad to have limited myself to only one.

\subsection{Evaluation}
The most important lesson in terms of evaluation and training I learned about
metrics and loss is that a model that is almost perfect (in terms of certain
metrics) may still produce meaningless results. A high Accuracy or a F1 value
may still mean that the desired results are not achieved. This is due to the
fact that these metrics only represent an abstract value and abstract from the
concrete situation. Therefore, it is always important -- even if sometimes
tedious -- to look at concrete examples in the evaluation and not just to stay
with the abstract metrics, as these can often be deceptive. In my case, as
described above, my model had a very high accuracy, because the generated
melodies were simply empty and therefore almost all notes were predicted
correctly. The fact that this output completely contradicted the original
approach is of course irrelevant for the model -- it does what it is told to do.
By changing the loss function, I was able to counteract this imbalance in the
data (to a certain extent). In any case, the insight remains that in future
projects I will pay even closer attention to the metrics used and the loss, and
that I will always check the results with concrete individual cases during the
evaluation and not rely merely on the simple metrics of the model.

\subsection{Hardware}
In addition to the aforementioned problem of empty melodies, I also had to
struggle in particular with the limitations of my hardware, which does NOT
include an Nvidia graphics card. To train an epoch merely with Mozart's sonatas
with my CPU takes me a little more than half an hour. However, in this field of
research, training over several hundred epochs is not uncommon. Because of this
limitation, I have only ever been able to carry out my experiments with even
smaller samples and a few epochs, which of course greatly distorts the results.
This problem can either be solved by reverting to a simpler model (which is
sometimes not possible) or to outsource the computation to a different machine,
e.g. cloud computing service.


\section{Workload}
\begin{table}[!h]
  \begin{center}
    \begin{tabular}{lrr}
      Task                                           & estimated & actual \\
      \hline
      \hline
      Dataset collection                             & 7         & 12     \\
      Exploring, analysing and preparing data        & 12        & 45     \\
      Designing and building an appropriate network  & 25        & 40     \\
      Training and fine-tuning that network          & 15        & 17     \\
      Building an application to present the results & 20        & 21     \\
      Writing the final report                       & 8         & 6      \\
      Preparing the presentation of your work        & 5         & 5      \\
      \hline
      Sum                                            & 92        & 147
    \end{tabular}
    \caption{Workload}
  \end{center}
\end{table}

Overall, I invested an above-average amount of time in the project. On the one
hand, this is due to the fact that I had an incredible amount of fun and
enjoyment with the project, since it encompasses the entire process from the
idea to the deployment and you can shape and follow the entire development.
Also, the fact that we were encouraged to structure the project cleanly and
neatly, to include automated tests and to generally orientate ourselves to the
current business workflow and not just deliver a Jupyter notebook, I found very
inspiring and justified for me to invest more time than planned.

On the other hand, and this is also reflected in the divergence between the
estimated and actual timeline, I especially underestimated the data preparation
and preprocessing. This is certainly due to the underlying data structure
(MIDI), which cannot be easily processed or used for machine learning -- many
preprocessing steps were required here. In addition, I was initially
overwhelmed by the sheer amount of options for representing MIDI data and
researched for a long time before deciding on one. In retrospect, it would have
made more sense to choose a simple approach from the beginning and implement it
as soon as possible to have a working pipeline. In general, this is one of the
main lessons I take away from the project: implement a simple pipeline
(baseline) quickly at the beginning and develop further based on it, instead of
starting with the biggest step and implementing everything perfectly (state of
the art) from the beginning.

The second major divergence between estimated and actual timeline is found in
designing and building an appropriate network. This is especially due to the
fact that I have never implemented a generative algorithm using RNNs before and
wanted to get to know the structure in more detail to create a kind of
blueprint for myself that I can use for further projects. In addition, I wanted
to work with Pytorch Lightning, which I have also never used before, as it
eliminates the need for a lot of boilerplate code. So the time difference in
the network designing process is especially also due to learning new framworks
and techniques as well as building a well documented blueprint and could have
been reduced for sure. However, it was worth it to me as it will save me
additional time in the future.

So in summary, the next time I would start with a reduced approach and work my
way up from there in small steps instead of doing everything in one go. This
not only has the advantage that you quickly have a baseline (and a
corresponding sense of achievement), but also that you can track the individual
steps and improvements more precisely and thus have a better comparison between
the models. Otherwise, I am very satisfied with my way of working and structure
and would also try to organize it accordingly well for further projects.



% \pagebreak
% What is the problem that you tried to solve?
% Why is it a problem?
% What is your solution?
% Why is it a solution? (And in particular, why is or isn’t deep learning a solution?)
% > Sequence awareness vs rule based!

% Additionally, it should cover:

% - main insights (what improved what)
% - what would I do differently next time

% The main take-aways and insights you gained from your project (e.g.,
% batch-normalization improved the results significantly, Adadelta worked much
% better than SGD on my data, annotating data works really good with tool X,
% setting up the pre-processing takes much more time than I expected, ...) If you
% would do the same project again, what - if anything - would you do differently?
% How much time did you spend on your project? How does the number compare to
% your initial estimate? If you underestimated any part, what were the reasons
% for this?

\printbibliography

\end{document}